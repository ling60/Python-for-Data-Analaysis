%\documentclass[preprint,12pt]{article}
\documentclass[11pt,fleqn]{book} % Default font size and left-justified equations

\usepackage[UTF8]{ctex}

% ********* Font definiton ************
\usepackage{t1enc} % as usual
%\usepackage[latin1]{inputenc} % as usual
\usepackage{times}		
%\usepackage{mathptmx}  	%mathematical fonts for use with times, 
%I encountered some problems using this package together with pdftex, which I was not able to resolve
\linespread{1.3} %1.5 line space

%% inline and display quotations
\usepackage{csquotes}

%%rorate the table
\usepackage{rotating}
%% the following commonds have been used in my other documents.
%% They may or may not be useful with APA6
\usepackage{graphicx}
%
%
% ********* Table layout **************
\usepackage{booktabs}	  	%design of table, has an excellent documentation


%\usepackage{lscape}			% make landscape table
\usepackage{rotating} %another choice of rotating the table

\usepackage{multirow} % multirows
\usepackage{array}
% Commands for use in tables
%\newcommand{\rr}{\raggedleft} % right justification
\newcolumntype{L}[1]{>{\raggedright\let\newline\\\arraybackslash\hspace{0pt}}m{#1}} % the following commands are for setting the spanning rows
\newcolumntype{C}[1]{>{\centering\let\newline\\\arraybackslash\hspace{0pt}}m{#1}}
\newcolumntype{R}[1]{>{\raggedleft\let\newline\\\arraybackslash\hspace{0pt}}m{#1}}
%\newcommand{\tn}{\tabularnewline} % New line command for use with \rr

\interfootnotelinepenalty=10000 % this prevents footnote spanning two pages

%\usepackage{rotating} % make landscape table

% ********* Caption Layout ************
\usepackage[font=small,skip=1pt]{caption} % allows special formating of the captions
%\usepackage[skip=2pt]{subcaption} %sub figs
%\addtolength{\subfigcapskip}{-0.2in}


\usepackage{varioref} % use this package to make cross referencing with page numbers


%%********** Equation ***********
\usepackage{amsmath} % allow equation  splited into two rows and hypothesis enviornment
\newtheorem{hyp}{Hypothesis} 

\makeatletter
\newcounter{subhyp} 
\let\savedc@hyp\c@hyp
\newenvironment{subhyp}
 {%
  \setcounter{subhyp}{0}%
  \stepcounter{hyp}%
  \edef\saved@hyp{\thehyp}% Save the current value of hyp
  \let\c@hyp\c@subhyp     % Now hyp is subhyp
  \renewcommand{\thehyp}{\saved@hyp\alph{hyp}}%
 }
 {}
\newcommand{\normhyp}{%
  \let\c@hyp\savedc@hyp % revert to the old one
  \renewcommand\thehyp{\arabic{hyp}}%
} 

%\usepackage[hyphens]{url}  # add line breaks to urls
%********** Enybeling Hyperlinks *******
%\usepackage[pdfborder=000,pdftex=true]{hyperref}% this enables jumping from a reference and table of content in the pdf file to its target
\usepackage[pdftex]{hyperref}
\usepackage[all]{hypcap} %let the firgure/table point to the position which is higher than caption
\hypersetup{
    %bookmarks=false,         % show bookmarks bar?
    unicode=false,          % non-Latin characters in Acrobatӳ bookmarks
    pdftoolbar=true,        % show Acrobatӳ toolbar?
    pdfmenubar=true,        % show Acrobatӳ menu?
    pdffitwindow=false,     % window fit to page when opened
    pdfstartview={FitH},    % fits the width of the page to the window
    pdftitle={My title},    % title
    pdfauthor={Author},     % author
    pdfsubject={Subject},   % subject of the document
    pdfcreator={Creator},   % creator of the document
    pdfproducer={Producer}, % producer of the document
    pdfkeywords={keywords}, % list of keywords
    pdfnewwindow=true,      % links in new window
    colorlinks=false,       % false: boxed links; true: colored links
    linkcolor=red,          % color of internal links
    citecolor=green,        % color of links to bibliography
    filecolor=magenta,      % color of file links
    urlcolor=cyan           % color of external links
}

%********** Citations **********
\usepackage{natbib} %let the reference use harvard style
\setcitestyle{aysep={,}}
%\bibliographystyle{agsm1}

\usepackage[T1]{fontenc}  % to display > and < correctly


% \pagestyle{plain} % on headers or footers on the first page

\begin{document}

%\title{A Non-Programmer Guide for Python for Data Analysis}
%\author{Ling Liu}
%\frontmatter
%\maketitle
%\tableofcontents
%\mainmatter
%
%\chapter{Data Analytica Models}\label{ch:models}
%In this lecture we will discuss how to use Python to build data analytic models.

\section{Statistical Models}
Statsmodels is probably the most common Python package for statistical models.
Statsmodels(http://www.statsmodels.org/stable/) is a Python module that provides classes and functions for the estimation of many different statistical models,
as well as for conducting statistical tests,
and statistical data exploration.
An extensive list of result statistics are available for each estimator.
The results are tested against existing statistical packages to ensure that they are correct. 

\subsection{Simple OLS}
在调用相关的package之前,我们需要先理解模型的意思,比如下面这个多元回归方程:

\[
y = \alpha + \beta_1X_1 + \beta_2X_2 + \varepsilon
\]

明白模型的意思,在于,我们要输入给模型的是什么类型的数据?想要的结果是什么?

通常而言,我们要做的,或者说常见的最小二乘法OLS想做的内容是拟合出一个最优化的模型。
这个模型能够较好的表达几个变量之间的关系。
我们是采用如下步骤来使用Statsmodels:
\begin{enumerate}
\item 处理数据$Xs \& Y(s)$ 为适合的数据类型。
\item 选择模型.如有需要,用``add\_constant()``添加constant.
\item 对模型做拟合得到模型的结果 ``model.fit()``
\item 获得模型的summary, ``results.summary()``
\item 如有需要从 results从获取各个参数
\end{enumerate}

具体做法可以参考:\url{https://www.statsmodels.org/dev/examples/index.html}。 里面包含了主要的统计模型的例子,包括time series的。
至于2/3SLS等模型无非是简单OLS的多次运行或变形,具体操作是相同的。


\section{Machine Learning Models}
Sci-kit learn 是Python主要的机器学习package。
与统计模型相类似的,甚至是更为重要的是,在跑模型之前,一定要先想清楚要选择的是什么模型?这个模型的基本要求是什么,输入和输出的数据类型是什么?

一般来说,具体做法如下:
\begin{enumerate}
\item 选择一个合适的模型
\item 设置模型的相关参数
\item 把数据整合为一个feature matrix和一个target vector. Scikit-Learn通常只接受 NumPy arrays, Pandas DataFrames, SciPy sparse matrices 几种数据类型
\item 调用fit()模型拟合
\item 把模型应用到新的数据上
\begin{itemize}
\item 对于有监督学习来说,我们一般调用predict()方法
\item 对无监督学习,我们则调用transform()或predict()
\end{itemize}

\end{enumerate}

具体实例,可以参考:\url{https://scikit-learn.org/stable/auto_examples/index.html}


%\bibliographystyle{model1b-num-names} % numbering reference
%\chapter*{Reference}
%\bibliography{reference}

\end{document}